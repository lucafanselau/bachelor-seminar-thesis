%%%%%%%%%%%%%%%%%%%%%%%%%%%%%%%%%%%%%%%%%%%%%%%%%%%%%%%%%%%% 
% This is the official template for theses and seminar papers from the Chair for Information Systems for Sustainable Society (IS3) at the University of Cologne

%
%PREAMBLE
%%%%%%%%%%%%%%%%%%%%%%%%%%%%%%%%%%%%%%%%%%%%%%%%%%%%%%%%%%%%%

\documentclass[a4paper, oneside, 12pt]{article}
\usepackage[utf8]{inputenc}
\usepackage[T1]{fontenc}
\usepackage{graphicx}
\usepackage{longtable}
\usepackage{hyperref}
\usepackage{caption}
\usepackage{amssymb}
\usepackage{amsmath}
\usepackage{todonotes}
\usepackage{booktabs}
\usepackage{tabularx}
\usepackage{rotating}

% set margins for double-sided printing
\usepackage[left=2.5cm, right=2.5cm, top=2.5cm, bottom=2.5cm, bindingoffset=1.5cm, head=15pt]{geometry} 
\usepackage{setspace}
\onehalfspacing
% set headers
\usepackage{fancyhdr}
\pagestyle{fancy}
\fancyhead{}
\fancyfoot{}
\fancyhead[L,RO]{\textsl{\leftmark}}
\fancyhead[R,LO]{\thesisauthor}
\fancyfoot[C]{\thepage}
\renewcommand{\headrulewidth}{0.4pt}
\renewcommand{\footrulewidth}{0pt}

% set APA citation style

\usepackage{apacite}
\usepackage[numbib,notlof,notlot,nottoc]{tocbibind}
\pagenumbering{gobble}

%%%%%%%%%%%%%%%%%%%%%%%%%%%%%%%%%%%%%%%%%%%%%%%%%%%%%%%%%%%%%
%THESIS Parameters 
%%%%%%%%%%%%%%%%%%%%%%%%%%%%%%%%%%%%%%%%%%%%%%%%%%%%%%%%%%%%%

\title{Using discrete event simulation to quantify performance of a car-sharing network with a heterogeneous shared fleet}

\newcommand{\thesisdate}{January 23rd, 2022}
\newcommand{\thesisauthor}{Luca Elias Fanselau} %input name
\newcommand{\studentID}{7369806} %input student ID
\newcommand{\thesistype}{Seminar Paper} % Set either to Bachelor or Master
\newcommand{\supervisor}{Univ.-Prof. Dr. Wolfgang Ketter}
\newcommand{\cosupervisor}{Muhammed Demircan}

%%%%%%%%%%%%%%%%%%%%%%%%%%%%%%%%%%%%%%%%%%%%%%%%%%%%%%%%%%%%%
%DOCUMENT
%%%%%%%%%%%%%%%%%%%%%%%%%%%%%%%%%%%%%%%%%%%%%%%%%%%%%%%%%%%%%

\begin{document}

%%%%%%%%%%%%%%%%%%%%%%%%%%%%%%%%%%%%%%%%%%%%%%%%%%%%%%%%%%%%%
%TITLE PAGE (Pre-defined, just change parameters above)
%%%%%%%%%%%%%%%%%%%%%%%%%%%%%%%%%%%%%%%%%%%%%%%%%%%%%%%%%%%%%
\input{Template/Title.tex}

%%%%%%%%%%%%%%%%%%%%%%%%%%%%%%%%%%%%%%%%%%%%%%%%%%%%%%%%%%%%%
%SOOA
%%%%%%%%%%%%%%%%%%%%%%%%%%%%%%%%%%%%%%%%%%%%%%%%%%%%%%%%%%%%%
\input{Template/SOOA.tex}

%%%%%%%%%%%%%%%%%%%%%%%%%%%%%%%%%%%%%%%%%%%%%%%%%%%%%%%%%%%%%
%ABSTRACT
%%%%%%%%%%%%%%%%%%%%%%%%%%%%%%%%%%%%%%%%%%%%%%%%%%%%%%%%%%%%%
\clearpage
\thispagestyle{empty}
\section*{Abstract}


In the rapidly growing market of car-sharing mobility, operational decision-making
can determine success or failure for car-sharing organizations. Predicting the impact
of these decisions analytically is a complex and high dimensional problem. Further complexity
comes from the heterogeneity of the vehicle fleet a customer is offered.
This paper proposes
a discrete event simulation that aims to provide quantitative metrics,
indicating the performance of the car-sharing networks and therefore providing a computationally
efficient way to analyze the effect of operational decisions. The simulation environment models a non-floating
one-way car-sharing system, equipped with a vehicle choice classifier that
is trained using existing car-sharing rental data. The classifier receives a customer request as input,
that is defined by socio-demographic factors as well as the planned route.
The proposed framework is then implemented for a specific car-sharing deployment in Berlin, Germany.
The dynamics of the model, in terms of the presented metrics,
are then analyzed in regard to the fleet size as well as the substitution effect, that
captures the choice uncertainty when selecting a specific vehicle type
that is typically present in real-world
decision-making. 



%%%%%%%%%%%%%%%%%%%%%%%%%%%%%%%%%%%%%%%%%%%%%%%%%%%%%%%%%%%%%
%TOC,TOF,TOT
%%%%%%%%%%%%%%%%%%%%%%%%%%%%%%%%%%%%%%%%%%%%%%%%%%%%%%%%%%%%%
\clearpage
\pagenumbering{Roman}
\tableofcontents
\clearpage
\listoffigures
\clearpage
\listoftables

\section*{List of Symbols}

\begin{tabularx}{\linewidth}{lX}
    $C$ & Parameter of the simulation model. Represents the maximum capacity of each vehicle class at each station \\
    $\alpha$ & Parameter of the simulation model. Represents the substitution effect. \\
    $\mathbb{C}$ & The set of all vehicle class choices \\
    $\mathbb{R}$ & The set of all real numbers \\
    $\mathbb{N}$ & The set of all natural numbers \\
    $cost$ & Function that maps the class and rental time to a price \\
    $\mathbb{X}$ & Set of socio-demographic categories that define an area \\
    $a$ & A specific area in regard to the categories $\mathbb{X}$ \\
    $\mathbb{S}$ & Set of all stations in the simulation \\
    $\mathbb{E}$ & Set of all connecting edges of the graph that is formed by $\mathbb{S}$ \\
    $L: \mathbb{E} \to \mathbb{R}$ & Distance mapping for each edge \\
    $\Delta_s$ & The delta function of a station. Indicates the number of vehicles of each type that is rented up until a specific simulation time. Always greater than $-C$\\
    $r$ & A customer request, containing a start station $s_{r, 0}$ and an end station $s_{r, 1}$ \\
    $D$ & A function that returns the set of all feasible vehicle choices for a request, with regard to the substitution effect \\
    $D'$ & Restricted subset of $D$ that checks availability of the vehicle type at $s_{r, 0}$ \\
    $URR$ & Unsatisfied Request Ratio \\
    $\pi$ & The total profit of the simulation stage \\
    $TD$ & The total distance driven during the simulation stage \\
\end{tabularx}


\clearpage

\pagenumbering{arabic}


%%%%%%%%%%%%%%%%%%%%%%%%%%%%%%%%%%%%%%%%%%%%%%%%%%%%%%%%%%%%%
%MAIN PART
%%%%%%%%%%%%%%%%%%%%%%%%%%%%%%%%%%%%%%%%%%%%%%%%%%%%%%%%%%%%%

% Introduction
\clearpage
\section{Introduction}
\label{sec:Intro}

Car sharing services have recently seen a rapid rise in popularity 
and valuations of the global car-sharing market expect a growth to
a total value of USD 6.5 billion by 2024, from just USD 1.1 billion in 2015 \shortcite[p.~1]{Emissions2018}. 
A primary reason for that is the high value-proposition a car-sharing service
can offer to its customers. This includes positive environmental impact in the form of
reduced CO2 emission by up to 312 kg CO2 / year per individual, as well as
lower individual mobility costs, compared to owning a vehicle \shortcite[p.~1525]{UlmEnv2011}. 
Together with the
broad adoption of smartphones, which, thought the use of custom apps, can
support the reservation and operation of shared vehicles, Car Sharing Organizations (CSOs)
can provide an appealing alternative to other public or private transportation measures.

These services can be categorized into free-floating, meaning vehicles of the fleet
can travel freely in a restricted area, and non-floating systems, where vehicles must travel
between discrete stations that typically offer a fixed amount of capacity. Additionally,
there is also a distinction between one-way and two-way car-sharing systems, the former
describes services where the user can travel from point A to point B, which he can choose
freely, while the latter expects the user to return the vehicle to the spot where it was
rented.

Operating large scale on-demand car-sharing networks tends to be difficult, since
deploying operational decisions is often bound to large expenses and, especially if the
system is already in use, cannot be done in a trial and error fashion. One way to test the feasibility of
seemingly optimal proposals quickly is to develop a simulation which tries to capture the real world
interaction of actors in a car sharing network to a sufficient degree. With this, an operator can 
quickly assess if the solution holds up to various real world restrictions that are characteristic
for car sharing networks, such as demand served or, in the case of an EV car-sharing platform, the vehicle
charge levels \shortcite[p.~224]{OptSimFramework}

The focus of this seminar thesis is firstly, to analyze existing literature on CSOs and especially
the use of simulation environments and the optimization problems that were solved using them.
Secondly, to use this information to design
and implement a discrete event simulation, that models a non-floating one-way on-demand
car-sharing service. The simulation is then equipped with a vehicle choice
classifier model, based on socio-demographic and request specific features,
to model the real world mobility dynamics of a car-sharing network.
The main objective of the proposed framework is to use metrics obtained by the simulation
to study the impact of fleet size and substitution effects on overall performance,
evaluated on the specific case of SHARE NOW in Berlin, Germany.


% Literature Review
\clearpage
\section{Literature Review}
\label{sec:lr}

A lot of pervious research on car-sharing and its impact has been conducted throughout
the last two decades. The focus here is to give an insight on relevant previous
research and extract important information and methods to this paper.

\cite{OptSimFramework}:
\begin{itemize}
    \item Integrated multi objective mixed integer linear programming opt.
    \item discrete event simulation
    \item models: station clustering, operations opt., personnell flow 
    \item Test feasibility in terms of: Electric vehicle battery
    \item Performance measures: Feasibility, Ratio of lost demand to total demand, number of relocations, number of personnel, relocations / personnell, vehicle utilization without / in relocation
    \item Clustering for reduced dimensionality of the problem 
    \item Hierarchical objective functions -> First quality of service over relocation cost
    \item Two types: 1 All information known a priori, 2 Requests appearing one at a time
    \item Res: Manage demand through use of insentives/disincentives 
\end{itemize}
    
\cite{Nourinejad2014}:
\begin{itemize}
    \item Propose a dynamic optimization-simulation model for one-way car-sharing networks, that acts based on
    online requests and tries to maximize system profit
    \item Sensitivity Analysis on fleet size and factors which impact the fleet size
    \item Event -> arrival of a new user
    \item 
\end{itemize}



% Methods
\clearpage
\section{Method}
\label{sec:Method}

In a heterogeneous fleet of vehicles that is typically deployed by a car-sharing operator,
having insights of the features determining the vehicle choice of a customer is key
to operate in an efficient manner. This includes features like availability, travel-time
and driving distance, but also socio-demographic features of the area in which the
car-sharing system is deployed. During this study we will develop a classifier which 
captures this relation and use it to analyze a inter-area connected network of stations
during a discrete event driven simulation phase to get insights on performance critical
metrics.

\subsection{Concepts}
\label{sub_sec:Method/Concepts}

Firstly a set of concepts is defined, that will be important in the further discussion.

\subsubsection{Vehicle Classes}
\label{sub_sec:Method/Concepts/Classes}

Since we are analyzing a fleet of heterogeneous vehicles, we define the set of vehicle
classes $\mathbb{C}$, to be the set of possible vehicle choices a customer can make. A
discrete example of these vehicle classes can be found in the case study later on. These
classes are also used to determine the pricing of a trip. An implementation of this
framework is expected to define a pricing function:
$$
C: \mathbb{C} \times  \mathbb{R} \to \mathbb{R}
$$
where the input is the rented class and the time traveled in minutes and the output is the
cost in local currency.

\subsubsection{Area}
\label{sub_sec:Method/Concepts/Area}

An area is an approximation of different socio-demographic distributions with regard to 
distinct classes. More specifically an area is defined in terms of the age function:
$$
\text{age}(a) = \{ P(d \in r): \forall r \in \mathbb{AG} \}
$$
where $\mathbb{AG} = \{ [0,17], [18,29], [30,49], [50,64], [65,\infty) \}$, $d \in D$ is the
actual age distribution for that particular area and $a$ is a valid area. The actual probability 
values will be determined empirically. Additional, due to the fact that the
Age Groups (AG) contains all possible values that $D$ can realize, the invariant 
$\sum_{p \in \text{age}(a)} p = 1$ holds true for every valid area. Another important 
characteristic is described by the income function of the area.
$$
\text{income}(a) = \{ P(i \in r): \forall r \in \mathbb{IG} \}
$$
where $\mathbb{IG} = \{ [0,899], [900, 1499], [1500, \infty) \}$ and $i \in I$ is the actual distribution
of income for the area $a$. This data will also be expected to be given and the same invariant holds true:
$\sum_{p \in \text{income}(a)} p = 1$.

\subsubsection{Station}
\label{sub_sec:Method/Concepts/Station}

The Station is an important concept for the simulation stage of the proposed framework. A station is always
inside a particular area, namely there exists a function $\text{area}(s)$ which returns the area that the 
station is placed in.

\subsection{Classification Model}
\label{sub_sec:Method/Class}

Using that information we can now define a classification model whose
main objective is to find a relation between the socio-demographic factors which describes an area
where a on-demand request is captured and the decision that would be made. We also want to determine the
probability of each class, so that we can later study the Substitution Effect % TODO: Ref to that
that captures a equality towards different decisions and can be controlled during the
simulation stage of the framework. We therefore propose a model whose objective is to
approximate the function $P(c \ | \ a)$ where $c \in \mathbb{C}$ and $a$ is a valid area.
Given this classifier, a prediction can be made which vehicle type is most likely to be chosen based
on a particular socio-demographic setting.

\subsection{Substitution Effect}
\label{sub_sec:Method/Substitution}

It is however noteworthy that this decision is only
partly driven by just those factors since especially in free-floating car-sharing services
other factors like closest car can dominate the actual decision process even against personal
preference. To capture this effect in a simplified form we will introduce a parameter $\alpha$
called the Substitution Effect, that captures the willingness to decide against the most likely
decision at random. Formally the set of all class decision $D$ of a request from a valid area $a$ and the 
Substitution effect $\alpha$ can then be defined as follows:
$$
  D(a, \alpha) = \{ c \ | \ c \in \mathbb{C} \land P(c \ | \ a) \ge \max\{ P(d \ | \ a) \ | \ d \in \mathbb{C} \} - \alpha \}
$$
and the actual decision as $d(a, \alpha) = \text{rand}(D(a, \alpha))$
where rand is a function which choses exactly one element of the set based on a random decision that
is not biased towards any of the items in the set. 

\subsection{Simulation}
\label{sub_sec:Method/Simulation}

Understanding the dynamics of a complex system like a car-sharing system analytically is exceptionally difficult.
One can however use Simulation to assess typical dynamics and get a quick insight on important performance metrics
like ratio of satisfied customers, total distance driven or total sales volume. Additionally the Simulation
environment should be easily extensible to enable further study on different network effects. A discrete-event
simulation is employed following the 



% Case Study
\clearpage
\section{Case Study Berlin SHARE NOW}
\label{sec:CaseStudy}

The methods described in Section \ref{sec:Method} are now applied to the specific case
of ShareNow in Berlin during a period between October 2019 and March 2020. Firstly
we will discuss ShareNow as a company and then later the dataset and the insights it
can deliver as well as the trained classifier and its application in the Simulation
environment we have described above. 

\subsection{SHARE NOW}

ShareNow is the world wide leading free-floating car-sharing service.
As a free-floating service the user can pick up an available car anywhere in a defined zone
and finish the trip anywhere in that zone. It is currently available in 16 european cities
with 11000 vehicles, with nearly 3000 electric vehicles. With about 3.4 million customers
it has a unprecedented set of car-sharing users \cite{ShareNowAboutUs}. It was founded as 
part of a larger a joint venture of the BMW Group and the Daimler AG, which also includes
services like PARK NOW, CHARGE NOW, REACH NOW and FREE NOW. Both firms brought in their
existing car-sharing solutions, namely Car2Go a subsidiary of the Daimler AG and DriveNow
a subsidiary of the BMW Group. The merge led to an increase in ease of use for customers
while the venture parties could secure the leading market share at many of europeans
largest car-sharing sites.


\subsection{Data sources}
\label{sub_sec:CaseStudy/Data}

This Case Study is primarily based on a dataset which includes 1.983.246 datapoints, each 
representing a trip made through SHARE NOWs service. Along other interesting fields each data point
contains the vehicle model of the rental as well as the location where the rental was started.
This data could then be joined with publicly available socio demographic data from Zensus 2011, a census
commissioned by the statistical federal office. This then captures the concept of an area (\ref{sub_sec:Method/Concepts/Area})
and can therefore be used to train a classification model on top of. Prior to that, the data points
that contained ill formed data with regards to our feature vector were removed and the 
demographic data was normalized to adhere to the invariant described during the definition of a
well formed area. In total 


\subsection{Fleet}
\label{sub_sec:CaseStudy/Fleet}

SHARE NOW uses an extensive fleet of vehicles for its car-sharing service according to their website. The data
for Berlin however indicates a slightly smaller set of available vehicles for that region. Throughout the
period of data collection a total of 3946 unique vehicles were tracked in the zone, with the following
distribution.

\begin{figure}[htbp]
  \centering
  \includegraphics[width=.5\linewidth]{./Figures/fleet.png}
  \caption{Number of unique vehicles by class}
  \label{fig:Fleet}
\end{figure}

As can be seen in Figure \ref{fig:Fleet} the number of vehicles of each type is far from equally distributed.
The dominant model in terms of number of vehicles is by far the smart fortwo with 1007 unique vehicles, 
offering a high value proposition
in terms of parking space and ease of use for urban commuting. A clear tendency towards smaller cars, such as smarts
or minis is visible in the distribution. The least common model on the other hand is the BMW X2 with just 29 unique vehicles,
further indicating the increased demand for smaller vehicles in this specific car-sharing system.

\begin{figure}[htbp]
  \centering
  \includegraphics[width=.5\linewidth]{./Figures/travels.png}
  \caption{Number of rentals by class}
  \label{fig:Rentals}
\end{figure}

As can be seen in Figure \ref{fig:Rentals} the actual realized demand of each model maps quite nicely
onto the number of unique vehicles in the fleet with only a few exception. Notably though the smart has
seen even more demand then indicated by the number of vehicles. While the sum of all mini vehicles in terms
of unique vehicles comes close to the number of smarts deployed, the number of unique rentals of a smart
is greater then the sum of rentals made with all other vehicle types.

The missing component however is the set of decidable classes. Similarly to the real world fleet deployment the
set of models is structured in terms of vehicle classes, namely $\mathbb{C} = \{ \text{XS}, \text{S}, \text{M}, \text{L} \}$.
Then each model was assigned to its matching class which can be found in table \ref{table:VehicleClasses}.
These classes play an important role in the operational management of the car-sharing fleet since they differ in terms of costs and therefore profit.
Another important aspect of these classes is that younger drivers, in the case of SHARE NOW in Germany drivers below an age of 21, are typically
restricted to the lowest class, which for this specific case contains just the model type smart fortwo and might explain
the extremely high demand for that model. Additionally we will define the cost function used in the simulation stage
of the framework based on the actual minute based pricing model of SHARE NOW. Although SHARE NOW also offers other tariffs we will focus one
the on-demand minute based charge system for this implementation \cite{ShareNowPricing}.
$$
C(c, t) = \begin{cases}
  t * 0.09, & \text{if $c$ = XS}\\
  t * 0.28, & \text{if $c$ = S}\\
  t * 0.31 + 0.99, & \text{if $c$ = M}\\
  t * 0.34 + 0.99, & \text{if $c$ = L}\\
 \end{cases}
$$


\subsection{Classifier}
\label{sub_sec:CaseStudy/Classifier}


\subsection{Simulation}
\label{sub_sec:CaseStudy/Simulation}

% Simulation
% \clearpage
\section{Simulation}
\label{sec:Simulation}

To test the dynamics of a car-sharing network using the vehicle choice model, that was derived in \ref{sec:ChoiceModel},
a discrete-event driven simulation framework will be designed and implemented in python. Following assumptions where
made

\subsection{Model Restrictions}
\label{subsec:Simulation/restrictions}


% Vehicle Choice Model
% \clearpage
\section{Vehicle Choice Model}
\label{sec:ChoiceModel}

Mathematical description of vehicle choice model

% Results & Discussion
\clearpage
\section{Results \& Discussion}
\label{sec:Results}

Could be done better:

Review feasibility of assumptions with larger datasets.

% Conclusion
\input{./Sections/Conclusion}

%%%%%%%%%%%%%%%%%%%%%%%%%%%%%%%%%%%%%%%%%%%%%%%%%%%%%%%%%%%%%
%APPENDICES
%%%%%%%%%%%%%%%%%%%%%%%%%%%%%%%%%%%%%%%%%%%%%%%%%%%%%%%%%%%%%


\appendix
\renewcommand*{\thesection}{\Alph{section}}\textbf{}

% APPENDIX A
\clearpage
\section{Appendix}
\label{app:A}

\begin{longtable}{ | l | p{12cm} |}
  \hline
  \textbf{Class} & \textbf{Vehicles} \\
  \hline
  XS & smart fortwo 3rd generation \\
  S & Mini 3 door, Mini 5 door, Mini Clubman, Mini Convertible,  Mini Countryman \\
  M & BMW 2er Active Tourer, BMW 1er, BMW I3, Mercedes-Benz A-Class, Mercedes-Benz B-Class, Mercedes-Benz GLA \\
  L & BMW 2er Cabrio, BMW X1, BMW X2 \\
  \hline

  \caption{Vehicle Classes in Case Study}
  \label{table:VehicleClasses}
\end{longtable}

\begin{longtable}{ | l | p{10cm} |}
  \hline
  Vehicle class \textbf{c} & $\text{cost}(\textbf{c}, t) = $ \\
  \hline
  XS & $t * 0.09$ \\
  S & $t * 0.28$ \\
  M & $t * 0.31 + 0.99$ \\
  L & $t * 0.34 + 0.99$ \\  
  \hline

  \caption{Cost function for vehicle classes}
  \label{table:CostFunction}
\end{longtable}

% \begin{equation}
% C(c, t) = \begin{cases}
%   t * 0.09, & \text{if $c$ = XS}\\
%   t * 0.28, & \text{if $c$ = S}\\
%   t * 0.31 + 0.99, & \text{if $c$ = M}\\
%   t * 0.34 + 0.99, & \text{if $c$ = L}\\
%  \end{cases}
%  \label{eq:CostFunction}
%  \caption{Cost function of vehicle classes}
% \end{equation}

\begin{longtable}{ | l | p{10cm} |}
  \hline
  \textbf{Range} & \textbf{Value} \\
  \hline
  $Z_{\text{AGE}}$ & $\{ [0, 18), [18, 30), [30, 50), [50, 65), [65, \infty)  \}$ \\
  $Z_{\text{MARITAL}}$ & $\{$ "Single", "Pairs", "Single Parents", "Parents with children", "Multiperson households" $\}$ \\
  \hline

  \caption{Socio demographic category ranges}
  \label{table:Ranges}
\end{longtable}

\begin{sidewaystable}
  \centering
\noindent
\begin{tabularx}{\textwidth}{|p{3cm}|*{5}X|*{5}X|}
  \hline
  {} & \multicolumn{5}{c|}{Age [Years]} & \multicolumn{5}{c|}{Marital Status} \\
  {} & 0-17 & 18-29 & 30-49 & 50-64 & Over 65 & Single & Pairs & Single Parents & Parents with children & Multi person household \\
  \textbf{District} & {}& {}& {}& {}& {}& {}& {}& {}& {}& {}\\\hline
  Charlottenburg-Wilmersdorf &           0.12 &        0.14 &        0.30 &        0.23 &          0.20 &             0.56 &  0.20 &           0.07 &                  0.14 &                  0.03 \\\hline
  Friedrichshain-Kreuzberg   &           0.14 &        0.23 &        0.40 &        0.14 &          0.09 &             0.54 &  0.15 &           0.07 &                  0.13 &                  0.11 \\\hline
  Lichtenberg                &           0.13 &        0.23 &        0.32 &        0.16 &          0.17 &             0.49 &  0.24 &           0.09 &                  0.14 &                  0.04 \\\hline
  Marzahn-Hellersdorf        &           0.05 &        0.07 &        0.34 &        0.27 &          0.27 &             0.25 &  0.64 &           0.02 &                  0.09 &                  0.00 \\\hline
  Mitte                      &           0.14 &        0.20 &        0.36 &        0.17 &          0.12 &             0.56 &  0.17 &           0.07 &                  0.13 &                  0.06 \\\hline
  Neukölln                   &           0.15 &        0.23 &        0.35 &        0.16 &          0.11 &             0.54 &  0.16 &           0.08 &                  0.13 &                  0.09 \\\hline
  Pankow                     &           0.15 &        0.18 &        0.46 &        0.11 &          0.10 &             0.55 &  0.17 &           0.08 &                  0.14 &                  0.06 \\\hline
  Reinickendorf              &           0.15 &        0.15 &        0.29 &        0.19 &          0.22 &             0.49 &  0.21 &           0.13 &                  0.12 &                  0.05 \\\hline
  Spandau                    &           0.12 &        0.17 &        0.28 &        0.24 &          0.19 &             0.56 &  0.21 &           0.08 &                  0.13 &                  0.02 \\\hline
  Steglitz-Zehlendorf        &           0.15 &        0.13 &        0.28 &        0.23 &          0.21 &             0.47 &  0.23 &           0.09 &                  0.18 &                  0.03 \\\hline
  Tempelhof-Schöneberg       &           0.14 &        0.15 &        0.33 &        0.21 &          0.17 &             0.54 &  0.19 &           0.08 &                  0.14 &                  0.04 \\\hline
  Treptow-Köpenick           &           0.16 &        0.13 &        0.31 &        0.12 &          0.28 &             0.46 &  0.21 &           0.10 &                  0.20 &                  0.03 \\\hline
\end{tabularx}
\captionof{table}{Districts in the Simulation
\label{table:Districts}}

\end{sidewaystable}

\begin{longtable}{ | l | c  c  c  c |}
  \hline
  \multicolumn{1}{|r|}{\textbf{Start}} & Pankow & Reinickendorf & Kreuzberg & Charlottenburg \\
  \textbf{Destination} & {} & {} & {} & {} \\\hline
  Pankow & 0 & 9.08 & 9.38 & 14.18 \\
  Reinickendorf & 9.08 & 0 & 12.39 & 8.46 \\
  Kreuzberg & 9.38 & 12.39 & 0 & 10.05 \\
  Charlottenburg & 14.18 & 8.46 & 10.05 & 0 \\
  
  \hline

  \caption{L function: Distance [in km] in Simulation}
  \label{table:Distance}
\end{longtable}


%%%%%%%%%%%%%%%%%%%%%%%%%%%%%%%%%%%%%%%%%%%%%%%%%%%%%%%%%%%%%
%BIBLIOGRAPHY
%%%%%%%%%%%%%%%%%%%%%%%%%%%%%%%%%%%%%%%%%%%%%%%%%%%%%%%%%%%%%

\clearpage
\renewcommand*{\thesection}{}\textbf{}

\bibliographystyle{apacite}
\bibliography{Bibliography.bib}


\end{document}
