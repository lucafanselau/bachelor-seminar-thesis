\clearpage
\section{Introduction}
\label{sec:Intro}

Car sharing services have recently seen a rapid rise in popularity 
and valuations of the global car-sharing market expect to grow to
a total value of USD 6.5 billion by 2024, from just USD 1.1 billion in 2015 \shortcite[p.~1]{Emissions2018}. 
A primary reason for that is the high value-proposition a car-sharing service
can offer to its customers. This includes positive environmental impact in the form of
reduced individual CO2 emission by up to 312 kg CO2 / year, as well as
lower individual mobility costs, compared to owning a vehicle \shortcite[p.~1525]{UlmEnv2011}. Together with the
broad adoption of smartphones equipped with apps, that typically
support the reservation and operation of shared vehicles, Car Sharing Organizations (CSOs)
can provide an appealing alternative to other public or private transportation measures.

These services can be categorized into free-floating, meaning vehicles of the fleet
can travel freely in a restricted area, and non-floating systems, where vehicles must travel
between discrete stations that typically offer a fixed amount of spaces. Additionally
there is also a distinction between one-way and two-way car-sharing systems, the former
describes services where the user can travel from point A to point B, which he can choose
freely, while the latter expects the user to return the vehicle to the spot where it was
rented.

Operating large scale on-demand car-sharing networks tends to be difficult, since
deploying operational decision is often bound to large expenses and, especially if the
system is already in use, can not be done in a trial and error fashion. One way to test the feasibility of
seemingly optimal proposals rapidly is to develop a simulation which tries to capture the real world
interaction of actors in a car sharing network to a sufficient degree. With this an operator can 
quickly asses if the solution holds up to various real world restrictions that are characteristic
for car sharing networks, such as demand served or, in the case of an EV car-sharing platform, the vehicle
charge levels \shortcite[p.~224]{OptSimFramework}

The focus of this seminar thesis is firstly, to analyse existing literature on CSOs and especially
the use of simulation environments and the optimization problems that were solved using them.
Secondly, to use that information to design
and implement a discrete event simulation, that models a non-floating one-way on-demand
car-sharing service. This simulation is then used to analyse the impacts of a vehicle
choice model on the optimal distribution of vehicle types per station and to get
insights on the dynamics of such a system. 
