\clearpage
\section{Literature Review}
\label{sec:lr}

A lot of pervious research on car-sharing and its impact has been conducted throughout
the last two decades. The focus of this section is to provide an overview of papers 
that convey relevant information to the simulation environment or the topic of vehicle
choice. For a broader literature review and further research pointers the reader can 
refer to dedicated papers on the topic such as \shortciteA{ILLGEN2019193} or \shortciteA{FERRERO2018501}.

\shortciteA{JIAN2017362} has proposed a Multiple discrete-continuous extreme value (MDCEV) modeling framework to predict 
a user vehicle type preference. Additionally, a simulation procedure is employed to evaluate the MDCEV
models, these results are then quantified by goodness-of-fit in regard to the data source.
The proposed model was developed to provide a useful tool for operators of car-sharing fleets.
They could prove a consistency of customer usage patterns regarding travel time, mileage and
monetary expenditure. Furthermore, an effect of various socio-demographic features such
as income, age, urban structure was discovered, indicating a statistical significance of these features
regarding the vehicle choice, that will also be discussed in this thesis.
As part of the MDCEV modeling procedure the satiation parameter was determined, that captures
the willingness to drive more with the current choice of vehicle type. It indicated that this
correlates with the usability, ease of driving and driving experience. Overall this paper
laid down a basis for the vehicle choice model in this paper and provided sufficient evidence that
a statistical important of the described features exists based on real world data.

\shortciteA{Nourinejad2014} research has provided a dynamic optimization-simulation model for 
one-way car-sharing networks, that acts based on
online requests with a reservation time and tries to maximize system profit. This is in contrast to the on-demand
framework that is developed throughout this thesis. Their research aims to study the effects
of fleet size and factors which impact fleet size and vehicle relocation cost. A benchmark
model is proposed that has complete knowledge and calculates an optimal solution, together with
the dynamic model, that could be used as a support tool for CSO operations. They also outlined
the benefits of a discrete event simulation framework from a computational standpoint. Findings
of the paper included a positive correlation of optimal fleet size and demand and that
the mean required fleet size per demand decreases with higher demand. They also were able to
quantify the operational effects of reservation times of 30 min, in contrast to the on-demand
model, can lead to a reduction in fleet size of up to 86\%. Additionally, they
concluded that the dynamic model, without total knowledge, converges to the results calculated
by the benchmark model and therefore verifies the feasibility of a discrete event driven
simulation for further studies. 

\shortciteA{PERBOLI2017} have used a simulation approach to study the effects of customized
tariff plans from the business model point of view. The proposed tool simulates different tariff plans
with regard to different profiles of car-sharing users, according to mobility needs and urban structure.
This is verified with the specific case of Turin, Italy. During the study, different car-sharing organizations
are characterized in terms of their business model canvas, a structured form of the companies main business
aspects. A separation of user profiles was made, into three distinct classes: commuters, professional users
and casual users, each differing in time of departure and range of distance. Findings for the specific
case of Turin include a favorability of dynamic pricing structures and concluded that private vehicle ownership
is beneficial, in regard to the best performing car-sharing service in Turin at that time, only after about 8000 km/year
for casual users and 10000 km/year for commuters or professional users. Additionally, a strong emphasis was 
made on the importance of simulation based models in car-sharing operation management.

\shortciteA{JOCHEM2020373} have conveyed a survey to study the effects of car-sharing
membership on individuals, with the specific case of SHARE NOW which is also the basis
for the simulation in this thesis. They specifically focused on the effects of such a 
membership on the car ownership. In an optimistic case one car-sharing car could replace
up to 20 regular vehicles. 

\shortciteA{OptSimFramework} developed an integrated multi objective mixed integer
linear programming optimization framework to optimize operational decision in regard to
vehicle and personnel relocation in a car-sharing system of electric vehicles with reservations. The output of 
the optimization phase was then tested in terms of feasibility by a discrete event simulation
stage. The determining factor of feasibility used is if the charge levels of electric vehicles
would be enough the satisfied a calculated schedule. The car-sharing system covered in this paper
was a non-floating one-way car-sharing system, as also covered in this thesis. Findings of this
study includes a near linear dependency of demand to amount of relocation necessary and also
verified the positive benefits from an operational standpoint that car-sharing services with reservation
times enable, which was also discovered by \shortciteA{Nourinejad2014}. Additionally, the benefit of controlling
demand by incentives/disincentives was discovered, as a higher amount of requests with the same 
resources could be served. This paper also used a clustering algorithm to group demand patterns into
general area, which closely mimics the concept of stations, which will be introduced in section \ref{sub_sec:Method/Concepts/Station},
this enabled the simulation to find results of real-world importance while reducing
computational complexity.
