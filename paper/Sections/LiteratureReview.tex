\clearpage
\section{Literature Review}
\label{sec:lr}

A lot of pervious research on car-sharing and its impact has been conducted throughout
the last two decades. The focus here is to give an insight on relevant previous
research and extract important information and methods to this paper.

\cite{OptSimFramework}:
\begin{itemize}
    \item Integrated multi objective mixed integer linear programming opt.
    \item discrete event simulation
    \item models: station clustering, operations opt., personnell flow 
    \item Test feasibility in terms of: Electric vehicle battery
    \item Performance measures: Feasibility, Ratio of lost demand to total demand, number of relocations, number of personnel, relocations / personnell, vehicle utilization without / in relocation
    \item Clustering for reduced dimensionality of the problem 
    \item Hierarchical objective functions -> First quality of service over relocation cost
    \item Two types: 1 All information known a priori, 2 Requests appearing one at a time
    \item Res: Manage demand through use of insentives/disincentives 
\end{itemize}
    
\cite{Nourinejad2014}:
\begin{itemize}
    \item Propose a dynamic optimization-simulation model for one-way car-sharing networks, that acts based on
    online requests and tries to maximize system profit
    \item Sensitivity Analysis on fleet size and factors which impact the fleet size
    \item Event -> arrival of a new user
    \item Parking and relocation optimization
    \item Metrics: Total revenue, total cost of relocation, fleet utilization, system reliability
    \item 209 parking spots
    \item TODO: Maybe more
\end{itemize}

\cite{JIAN2017362}:
\begin{itemize}
    \item Predict user vehicle type preference using a Multiple discrete-continuous extreme value (MDCEV) modeling framework. 
    \item Simulation procedure to obain performance metrics: RMSE, correct Ratio
    \item Useful for operators
    \item Focusing on travel time, mileage and expenditure as main features, that effect vehicle choice and usage
    \item High income -> not utility vehicles (since carsharing leizure \& business) and Luxury
    \item low income -> "special purposes" like moving
    \item Urban structure -> smaller
    \item Expenditure seems to be the ruling factor (by a bit)
\end{itemize}


