\section{Conclusion}
\label{sec:Conclusion}

This research aimed to develop a discrete event simulation and integrate a classification model
to get measurable insights into the mobility dynamics of a car-sharing system. By introducing a
simulation framework and defining operationally important performance metrics, a clear
positive dependency of overall performance to the value of the Substitution effect and the
fleet capacity has been identified. This has proven the ability to use simulation environments, 
sourced with real world data, to gain insights applicable to real world scenarios and guide
operational decision-making. Therefore, an answer to the main research question proposed in Section \ref{sec:Intro},
about the impact of the parameters $\alpha$, the Substitution effect, and $C$, the capacity,
has been found, and a simulation model has been designed 
that can be used as a foundation to perform additional research in the field of large
scale car-sharing system.


As outlined in Section \ref{sub_sec:Results/Discussion} the
proposed framework has multiple aspects that could be improved in further research but these
enhancements would
exceed the scope of this thesis. Overall the method of discrete event simulation has shown to be a
good fit to model this complex subject and provide a computationally efficient way of
quantitative analysis of the environment. To increase the applicability of the model for real-world
usage some as the assumptions, such as the equal spacial distribution of customer request should be
revisited.