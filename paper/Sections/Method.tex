\clearpage
\section{Method}
\label{sec:Method}

In a heterogeneous fleet of vehicles that is typically deployed by a car-sharing operator,
having insights of the features determining the vehicle choice of a customer is key
to operate in an efficient manner. This includes features like availability, travel-time
and driving distance, but also socio-demographic features of the area in which the
car-sharing system is deployed. During this study we will develop a classifier which 
captures this relation and use it to analyze a inter-area connected network of stations
during a discrete event driven simulation phase to get insights on performance critical
metrics.

\subsection{Concepts}
\label{sub_sec:Method/Concepts}

Firstly a set of concepts is defined, that will be important in the further discussion.

\subsubsection{Vehicle Classes}
\label{sub_sec:Method/Concepts/Classes}

Since we are analyzing a fleet of heterogeneous vehicles, we define the set of vehicle
classes $\mathbb{C}$, to be the set of possible vehicle choices a customer can make. A
discrete example of these vehicle classes can be found in the case study later on. These
classes are also used to determine the pricing of a trip. An implementation of this
framework is expected to define a pricing function:
$$
C: \mathbb{C} \times  \mathbb{R} \to \mathbb{R}
$$
where the input is the rented class and the time traveled in minutes and the output is the
cost in local currency.

\subsubsection{Area}
\label{sub_sec:Method/Concepts/Area}

An area is an approximation of different socio-demographic distributions with regard to 
distinct classes. More specifically an area is defined in terms of the age function:
$$
\text{age}(a) = \{ P(d \in r): \forall r \in \mathbb{AG} \}
$$
where $\mathbb{AG} = \{ [0,17], [18,29], [30,49], [50,64], [65,\infty) \}$, $d \in D$ is the
actual age distribution for that particular area and $a$ is a valid area. The actual probability 
values will be determined empirically. Additional, due to the fact that the
Age Groups (AG) contains all possible values that $D$ can realize, the invariant 
$\sum_{p \in \text{age}(a)} p = 1$ holds true for every valid area. Another important 
characteristic is described by the income function of the area.
$$
\text{income}(a) = \{ P(i \in r): \forall r \in \mathbb{IG} \}
$$
where $\mathbb{IG} = \{ [0,899], [900, 1499], [1500, \infty) \}$ and $i \in I$ is the actual distribution
of income for the area $a$. This data will also be expected to be given and the same invariant holds true:
$\sum_{p \in \text{income}(a)} p = 1$.

\subsubsection{Station}
\label{sub_sec:Method/Concepts/Station}

The Station is an important concept for the simulation stage of the proposed framework. A station is always
inside a particular area, namely there exists a function $\text{area}(s)$ which returns the area that the 
station is placed in.

\subsection{Classification Model}
\label{sub_sec:Method/Class}

Using that information we can now define a classification model whose
main objective is to find a relation between the socio-demographic factors which describes an area
where a on-demand request is captured and the decision that would be made. We also want to determine the
probability of each class, so that we can later study the Substitution Effect % TODO: Ref to that
that captures a equality towards different decisions and can be controlled during the
simulation stage of the framework. We therefore propose a model whose objective is to
approximate the function $P(c \ | \ a)$ where $c \in \mathbb{C}$ and $a$ is a valid area.
Given this classifier, a prediction can be made which vehicle type is most likely to be chosen based
on a particular socio-demographic setting.

\subsection{Substitution Effect}
\label{sub_sec:Method/Substitution}

It is however noteworthy that this decision is only
partly driven by just those factors since especially in free-floating car-sharing services
other factors like closest car can dominate the actual decision process even against personal
preference. To capture this effect in a simplified form we will introduce a parameter $\alpha$
called the Substitution Effect, that captures the willingness to decide against the most likely
decision at random. Formally the set of all class decision $D$ of a request from a valid area $a$ and the 
Substitution effect $\alpha$ can then be defined as follows:
$$
  D(a, \alpha) = \{ c \ | \ c \in \mathbb{C} \land P(c \ | \ a) \ge \max\{ P(d \ | \ a) \ | \ d \in \mathbb{C} \} - \alpha \}
$$
and the actual decision as $d(a, \alpha) = \text{rand}(D(a, \alpha))$
where rand is a function which choses exactly one element of the set based on a random decision that
is not biased towards any of the items in the set. 

\subsection{Simulation}
\label{sub_sec:Method/Simulation}

Understanding the dynamics of a complex system like a car-sharing system analytically is exceptionally difficult.
One can however use Simulation to assess typical dynamics and get a quick insight on important performance metrics
like ratio of satisfied customers, total distance driven or total sales volume. Additionally the Simulation
environment should be easily extensible to enable further study on different network effects. A discrete-event
simulation is employed following the 

