\clearpage
\section{Case Study Berlin SHARE NOW}
\label{sec:CaseStudy}

The methods described in Section \ref{sec:Method} are now applied to the specific case
of ShareNow in Berlin during a period between October 2019 and March 2020. Firstly
we will discuss ShareNow as a company and then later the dataset and the insights it
can deliver as well as the trained classifier and its application in the Simulation
environment we have described above. 

\subsection{SHARE NOW}

ShareNow is the world wide leading free-floating car-sharing service.
As a free-floating service the user can pick up an available car anywhere in a defined zone
and finish the trip anywhere in that zone. It is currently available in 16 european cities
with 11000 vehicles, with nearly 3000 electric vehicles. With about 3.4 million customers
it has a unprecedented set of car-sharing users \cite{ShareNowAboutUs}. It was founded as 
part of a larger a joint venture of the BMW Group and the Daimler AG, which also includes
services like PARK NOW, CHARGE NOW, REACH NOW and FREE NOW. Both firms brought in their
existing car-sharing solutions, namely Car2Go a subsidiary of the Daimler AG and DriveNow
a subsidiary of the BMW Group. The merge led to an increase in ease of use for customers
while the venture parties could secure the leading market share at many of europeans
largest car-sharing sites.


\subsection{Data sources}
\label{sub_sec:CaseStudy/Data}

This Case Study is primarily based on a dataset which includes 1.983.246 datapoints, each 
representing a trip made through SHARE NOWs service. Along other interesting fields each data point
contains the vehicle model of the rental, the location where the rental was started as well as the distance of the trip.
This data was then joined with publicly available socio-demographic data from Zensus 2011, a census
commissioned by the statistical federal office. 


\subsection{Fleet}
\label{sub_sec:CaseStudy/Fleet}

SHARE NOW uses an extensive fleet of vehicles for its car-sharing service according to their website. The data
for Berlin however indicates a slightly smaller set of available vehicles for that region. Throughout the
period of data collection a total of 3946 unique vehicles were tracked in the zone, with the following
distribution.

\begin{figure}[htbp]
  \centering
  \includegraphics[width=.5\linewidth]{./Figures/fleet.png}
  \caption{Number of unique vehicles by class}
  \label{fig:Fleet}
\end{figure}

As can be seen in Figure \ref{fig:Fleet} the number of vehicles of each type is far from equally distributed.
The dominant model in terms of number of vehicles is by far the smart fortwo with 1007 unique vehicles, 
offering a high value proposition
in terms of parking space and ease of use for urban commuting. A clear tendency towards smaller cars, such as smarts
or minis is visible in the distribution. The least common model on the other hand is the BMW X2 with just 29 unique vehicles,
further indicating the increased demand for smaller vehicles in this specific car-sharing system.

\begin{figure}[htbp]
  \centering
  \includegraphics[width=.5\linewidth]{./Figures/travels.png}
  \caption{Number of rentals by class}
  \label{fig:Rentals}
\end{figure}

As can be seen in Figure \ref{fig:Rentals} the actual realized demand of each model maps quite nicely
onto the number of unique vehicles in the fleet with only a few exception. Notably though the smart has
seen even more demand then indicated by the number of vehicles. While the sum of all mini vehicles in terms
of unique vehicles comes close to the number of smarts deployed, the number of unique rentals of a smart
is greater then the sum of rentals made with all other vehicle types.

The missing component however is the set of decidable classes. Similarly to the real world fleet deployment the
set of models is structured in terms of vehicle classes, namely $\mathbb{C} = \{ \text{XS}, \text{S}, \text{M}, \text{L} \}$.
Then each model was assigned to its matching class which can be found in table \ref{table:VehicleClasses}.
These classes play an important role in the operational management of the car-sharing fleet since they differ in terms of costs and therefore profit.
Another important aspect of these classes is that younger drivers, in the case of SHARE NOW in Germany drivers below an age of 21, are typically
restricted to the lowest class, which for this specific case contains just the model type smart fortwo and might explain
the extremely high demand for that model. Additionally we will define the cost function used in the simulation stage
of the framework based on the actual minute based pricing model of SHARE NOW. Although SHARE NOW also offers other tariffs we will focus one
the on-demand minute based charge system for this implementation \cite{ShareNowPricing}.
$$
C(c, t) = \begin{cases}
  t * 0.09, & \text{if $c$ = XS}\\
  t * 0.28, & \text{if $c$ = S}\\
  t * 0.31 + 0.99, & \text{if $c$ = M}\\
  t * 0.34 + 0.99, & \text{if $c$ = L}\\
 \end{cases}
$$

\subsection{Data preparation}
\label{sub_sec:CaseStudy/Preparation}

Before one could train the actual classifier the data had to be prepared to adhere to the from
required by the framework. Using the zensus data an area description for each request was
build where the set of categories to be analyzed includes the age distribution as well as the martial status.
$\mathbb{X} = \{ \text{AGE}, \text{MARTIAL} \}$. Then ranges for each were defined as can be seen in Table \ref{table:Ranges}.
Subsequently rows with missing of malformed data were removed.

In early training iterations another important imbalance in the dataset was discovered. Classification models
trained on the data had a tendency to favor the "XS" class since there were far more rentals done with
vehicles from that class. To counteract that, since the set of features was missing an indicator for availability
of vehicles the data for each class was shuffled and truncated in such a way that
$\frac{nv(c_a)}{nv(c_b)} = \frac{nd(c_a)}{nd(c_b)} \ \forall c_a, c_b \in \mathbb{C}$, where $nv: \mathbb{C} \to \mathbb{N}$ is a function
that returns the number of vehicles of a specific category and $nd: \mathbb{C} \to \mathbb{N}$ is a function that returns the
amount of datapoints in the training set for each category. Leading to a dataset where the decision for a vehicle
should not be dominated by the amount of vehicles of that class.

\subsection{Classifier}
\label{sub_sec:CaseStudy/Classifier}

Resulting in a dataset that mapped the framework definition of the customer request onto 


\subsection{Simulation}
\label{sub_sec:CaseStudy/Simulation}