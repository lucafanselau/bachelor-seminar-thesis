\clearpage
\section{Vehicle Choice Model}
\label{sec:ChoiceModel}

Mathematical description of vehicle choice model

% Maybe a model that calculates if a vehicle will be rented
Palette: https://flatuicolors.com/palette/ru


\subsection{Data source}
\label{sub_sec:DataSource}

- Dataset 4gb
- Zensus data


\subsection{Data preparation}
\label{sub_sec:DataPreparation}

- Huge bias towards smart fourtwo
- Missing income
- 


\begin{itemize}
  \item House hold -> Size relation
  \item College to size -> No real thingy
  % \item Dataset: https://www.kaggle.com/steventaylor11/stated-preferences-for-car-choice
\end{itemize}

Features:

We propose a person, which is defined by a feature vector. From now on a person strictly refers to a specific realization 
of these features unless mentioned otherwise. 

\begin{longtable}{l | c}
  \caption{Features}
  \label{table:features}
  \\
  \textbf{Label} & \textbf{Description} \\
  \hline
  age & The current age of this person, must be > 18 years. unbounded otherwise \\
  hss & Household size\\
  college & 1 (or true) if the person had college education, 0 (or false) otherwise \\
  income & Monthly income, before taxes \\
  comm & Commuting distance (daily)
\end{longtable}

Scoring algorithm with the following "decisions":

Early opt out case: $age < 21$: Typically only the XS category is allowed to be driven by persons under the age of 21.
Otherwise we will employ a scoring algorithm which, where we calculate a score $s(o)$ where 
$o \in \{ \textrm{XS}, \textrm{S}, \textrm{M}, \textrm{L} \} = C$. The score is based on various sub-scores, namely:
$$
s_0(o) = 
\begin{cases}
  1 - P(\text{hss} > 2 \land \text{category} = o), & \text{if}\ \text{hss}\ \le 2 \\
  P(\text{hss} > 2 \land \text{category} = o), & \text{otherwise}
\end{cases}
$$
where $P(\text{hss} > 2 \land \text{category} = o) \ \forall o \in C$ is determined empirically by using the data source 
mentioned in \ref{sub_sec:DataSource} and can be found in the Appendix\todo{Add that to the Appendix}. Another important 
scoring mechanism depictures a correlation between commuting distance and the size an can be similarily described as:
$$
s_1(o) = 
\begin{cases}
  1 - P(\text{comm} > 2 \land \text{category} = o), & \text{if}\ \text{hss}\ \le 2 \\
  P(\text{comm} > 2 \land \text{category} = o), & \text{otherwise}
\end{cases}
$$

